\subsection{Generic pattern detector}
\label{subsec:genericpd}


% ----------------------- paths to graphics ------------------------



% ----------------------- contents from here ------------------------
% 

The purpose of a generic pattern detector is to serve as an interface in the pattern detection and compression learning processes. Given a sample of data and its schema, an instance of a generic pattern detector searches for the presence of the pattern it is specialized in.  For each column that matches the pattern, it outputs metrics and metadata that will be further used in the learning phase and during compression. This generic design allows new pattern detectors to be easily tested and integrated into the system without modifying other parts of it (e.g. the learning or compression processes). This section presents the characteristics of the generic pattern detector interface, while specialized implementations of it are described in the next sections.

The pattern detection process works in 3 phases: Phase-1: \textit{initialization}---initializes the pattern detector and creates the data structures that will be used in the next phases; Phase-2: \textit{scanning}---scans the sample of data and gathers information and metrics about values in the sample. Phase-3: \textit{evaluation}---aggregates the information gathered in the \textit{scanning} phase and produces an evaluation result.

\textbf{Parameters.} A pattern detector takes as input the following parameters:\\
a) \textit{columns}: id, name and datatype for each column\\
b) \textit{compression tree}: the compression tree built so far. Used in the \textit{select\_column} method\\
c) \textit{detection log}: history of the pattern detection process. Contains information about which pattern detectors were evaluated on each column and what was the result. Used in the \textit{select\_column} method.\\
Although most pattern detectors work on individual columns, some may work on multiple columns (e.g. \nameref{subsec:pd:columncorrelation}). For this reason we designed the generic pattern detector to search for patterns at the table level instead of individual columns.

\textbf{Methods.} A pattern detector must implement the following methods:\\
a) \textit{select\_column}: called in the \textit{initialization} phase---determines whether a given column will be processed by the pattern detector. The decision process is based on: 1) data type (e.g. string-specific patterns do not apply to numeric columns); 2) path in the compression tree that led to the creation of the column (e.g. \nameref{subsec:pd:columncorrelation} only applies to columns that are output of a \nameref{subsec:pd:dict} compression node); 3) history of the pattern detection process (e.g. do not evaluate a pattern detector on a column if it was already evaluated in a previous step). Column selection rules are listed in the section of each pattern detector.\\
b) \textit{feed\_tuple}: repeatedly called in the \textit{scanning} phase---processes a tuple. Data is fed to the pattern detector one tuple at a time. This method extracts information from the tuple that will be later used in the \textit{evaluate} method.\\
c) \textit{evaluate}: called in the \textit{evaluation} phase---aggregates the information extracted from the tuples fed so far and outputs the outcome of applying the pattern to the columns. See output details below.

\textbf{Output.} The result of evaluating a pattern detector on a sample of data is a list of \textit{(expression node, evaluation result)} tuples.\\
a) \textit{expression node}: describes how one or more \textit{input columns} are transformed into one or more \textit{output columns} by applying an \textit{operator} (see detailed description in \ref{ch:exprlang}~\nameref{ch:exprlang}). An important part here is the operator metadata that will be used to evaluate the \textit{expression node} in the (de)compression process (e.g. the operator metadata for the \nameref{subsec:pd:dict} pattern detector is the dictionary/map object).\\
b) \textit{evaluation result}: contains information used in the learning process and describes how well the \textit{input columns} of the \textit{expression node} match the pattern. It contains the following information: 1) \textit{coverage} (percentage of rows that the pattern applies to; i.e. \(1 - exception\_ratio\)); 2) \textit{row\_mask} (bitmap indicating the rows that the pattern applies to); 3) other pattern-specific evaluation results (e.g. correlation coefficient for column correlation).

Each \textit{(expression node, evaluation result)} tuple represents a possibility of applying the pattern on a subset of the columns. The same column may be present in multiple tuples since there may be more than one option of applying the pattern to it (e.g. the \nameref{subsec:pd:charsetsplit} pattern detector can split a column in 2 ways, because there are 2 dominant character set patterns on the column; see \ref{subsec:pd:charsetsplit}~\nameref{subsec:pd:charsetsplit} for more details). The \textit{row\_masks} for such a column may either be mutually exclusive or partially overlap. A pattern detector only outputs results for columns that match the pattern and are likely to produce good results in the compression process according to a pattern-specific estimator (e.g. \nameref{subsec:pd:dict} pattern detector only outputs tuples for columns that are dictionary compressible; see \ref{subsec:pd:dict}~\nameref{subsec:pd:dict} for more details).

\textbf{Operators.} Each pattern detector provides a \textit{compression operator} and a \textit{decompression operator}. They are used in the compression and respectively decompression phase to evaluate the nodes in the expression trees. The generic operator is described in \ref{ch:exprlang}~\nameref{ch:exprlang} while the pattern-specific implementations can be found in the corresponding section of each pattern detector.

% TODO-1: image describing the input and output of the generic pattern detector

% ---------------------------------------------------------------------------
% ----------------------- end of thesis sub-document ------------------------
% ---------------------------------------------------------------------------